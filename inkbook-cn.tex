
% !TeX encoding = UTF-8
\documentclass[en]{inkbook}

\title{\textsf{inkbook}:一个充满墨香的书籍模板}
\author{Moyan Liang\thanks{\aka{梁莫言}.} \and Dapeng Zhang\thanks{\aka{张大鹏}.}}
\titheader{使用手册}
\equote{\begin{equation*}
    \int_a^b f(x)\mathd x =\mathcal{F}(x)\Big|_a^b. 
\end{equation*}}
\usepackage{lipsum}

\begin{document}
    \maketitle
    \tableofcontents
    \part{使用手册}
    
    \chapter{如何使用}
    \section{引言}
    这是在张大鹏排版的\textit{经典力学讲义}的基础上整理而成的模板.使用了C\TeX{}来显示中文.使用\textsf{mdframed}宏包来改写定理环境.
    \section{引用 \textsf{inkbook}}
    \begin{lstlisting}[language=TeX]
        \documentclass[(lang=)cn/en]{inkbook}
    \end{lstlisting}
    来引用\textsf{inkbook}.
    \section{定理环境}
    \begin{theorem}
        这是\textsf{theorem}环境.
    \end{theorem}
    \textsf{inkbook}提供了两种引理环境,一种适用于证明环境内,一种适用于普通环境内.
    \begin{fancylemma}
        这是适用于普通环境内的\textsf{fancylemma}环境.
    \end{fancylemma}
    \begin{lemma}
        这是适用于证明环境内的\textsf{lemma}环境.
    \end{lemma}
    \begin{proposition}
        这是\textsf{proposition}环境.
    \end{proposition}
    \begin{conjecture}
        这是\textsf{conjecture}环境.
    \end{conjecture}
    \begin{corollary}
        这是\textsf{corollary}环境.
    \end{corollary}
    \begin{figure}
        \begin{lstlisting}[language=TeX]
            \newtheorem{example}{Example}[chapter]
            \newtheorem{question}{Question}[section]
            \newtheorem*{solution}{Solution}
            \newblocktheorem[section]{theorem}{Theorem}
            \newblocktheorem[section]{fancylemma}{Lemma} %
            \newtheorem{lemma}{Lemma}[section]
            \newblocktheorem[section]{proposition}{Proposition} %
            \newblocktheorem{corollary}{Corollary} %
            \newblocktheorem{law}{Law}
            \newblocktheorem{definition}{Definition}[section] %
            \newblocktheorem{conjecture}{conjecture}[section] %
            \theoremstyle{remark} %
            \newtheorem*{remark}{\normalfont\bfseries Remark} %
            \newtheorem*{note}{\normalfont\bfseries Note} %
            \newtheorem{case}{\normalfont\bfseries Case} %
            \renewcommand*{\proofname}{\normalfont\bfseries\color{black}Proof} %
        \end{lstlisting}
        \caption{定义摘录}
    \end{figure}

    \begin{proof}
        \begin{equation}
        \int_a^b f(x)\mathd x =\mathcal{F}(x)\Big|_a^b. 
    \end{equation}
    \end{proof}
    \chapter{如何使用}

    \appendix

    \chapter{附录1}

    \section{附录1.1}

    \lipsum[1-5]

    \section{附录1.2}

    \lipsum[1-5]

    \chapter{附录2}

    \section{附录2.2}

    \lipsum[2]
    
\end{document}